\documentclass[journal, a4paper]{IEEEtran}
\renewcommand\thesection{\arabic{section}}
\renewcommand\thesubsection{\thesection.\arabic{subsection}}
\renewcommand\thesubsubsection{\thesubsection.\arabic{subsubsection}}
\usepackage{listings}             % Include the listings-package
\usepackage{ textcomp }
\lstset{language=Python}          % Set your language (you can change the language for each code-block optionally)
\lstdefinestyle{customc}{
  belowcaptionskip=1\baselineskip,
  breaklines=true,
  frame=L,
  xleftmargin=\parindent,
  language=Python,
  showstringspaces=false,
  basicstyle=\footnotesize\ttfamily,
  %keywordstyle=\bfseries\color{green!40!black},
  %commentstyle=\itshape\color{purple!40!black},
  %identifierstyle=\color{blue},
  stringstyle=\color{orange},
}

\lstdefinestyle{customasm}{
  belowcaptionskip=1\baselineskip,
  frame=L,
  xleftmargin=\parindent,
  basicstyle=\footnotesize\ttfamily,
  commentstyle=\itshape\color{purple!40!black},
}

\lstset{escapechar=@,style=customc}

% some very useful LaTeX packages include:

%\usepackage{cite}      % Written by Donald Arseneau
                        % V1.6 and later of IEEEtran pre-defines the format
                        % of the cite.sty package \cite{} output to follow
                        % that of IEEE. Loading the cite package will
                        % result in citation numbers being automatically
                        % sorted and properly "ranged". i.e.,
                        % [1], [9], [2], [7], [5], [6]
                        % (without using cite.sty)
                        % will become:
                        % [1], [2], [5]--[7], [9] (using cite.sty)
                        % cite.sty's \cite will automatically add leading
                        % space, if needed. Use cite.sty's noadjust option
                        % (cite.sty V3.8 and later) if you want to turn this
                        % off. cite.sty is already installed on most LaTeX
                        % systems. The latest version can be obtained at:
                        % http://www.ctan.org/tex-archive/macros/latex/contrib/supported/cite/

\usepackage{graphicx}   % Written by David Carlisle and Sebastian Rahtz
                        % Required if you want graphics, photos, etc.
                        % graphicx.sty is already installed on most LaTeX
                        % systems. The latest version and documentation can
                        % be obtained at:
                        % http://www.ctan.org/tex-archive/macros/latex/required/graphics/
                        % Another good source of documentation is "Using
                        % Imported Graphics in LaTeX2e" by Keith Reckdahl
                        % which can be found as esplatex.ps and epslatex.pdf
                        % at: http://www.ctan.org/tex-archive/info/

%\usepackage{psfrag}    % Written by Craig Barratt, Michael C. Grant,
                        % and David Carlisle
                        % This package allows you to substitute LaTeX
                        % commands for text in imported EPS graphic files.
                        % In this way, LaTeX symbols can be placed into
                        % graphics that have been generated by other
                        % applications. You must use latex->dvips->ps2pdf
                        % workflow (not direct pdf output from pdflatex) if
                        % you wish to use this capability because it works
                        % via some PostScript tricks. Alternatively, the
                        % graphics could be processed as separate files via
                        % psfrag and dvips, then converted to PDF for
                        % inclusion in the main file which uses pdflatex.
                        % Docs are in "The PSfrag System" by Michael C. Grant
                        % and David Carlisle. There is also some information
                        % about using psfrag in "Using Imported Graphics in
                        % LaTeX2e" by Keith Reckdahl which documents the
                        % graphicx package (see above). The psfrag package
                        % and documentation can be obtained at:
                        % http://www.ctan.org/tex-archive/macros/latex/contrib/supported/psfrag/

%\usepackage{subfigure} % Written by Steven Douglas Cochran
                        % This package makes it easy to put subfigures
                        % in your figures. i.e., "figure 1a and 1b"
                        % Docs are in "Using Imported Graphics in LaTeX2e"
                        % by Keith Reckdahl which also documents the graphicx
                        % package (see above). subfigure.sty is already
                        % installed on most LaTeX systems. The latest version
                        % and documentation can be obtained at:
                        % http://www.ctan.org/tex-archive/macros/latex/contrib/supported/subfigure/

\usepackage{url}        % Written by Donald Arseneau
                        % Provides better support for handling and breaking
                        % URLs. url.sty is already installed on most LaTeX
                        % systems. The latest version can be obtained at:
                        % http://www.ctan.org/tex-archive/macros/latex/contrib/other/misc/
                        % Read the url.sty source comments for usage information.

%\usepackage{stfloats}  % Written by Sigitas Tolusis
                        % Gives LaTeX2e the ability to do double column
                        % floats at the bottom of the page as well as the top.
                        % (e.g., "\begin{figure*}[!b]" is not normally
                        % possible in LaTeX2e). This is an invasive package
                        % which rewrites many portions of the LaTeX2e output
                        % routines. It may not work with other packages that
                        % modify the LaTeX2e output routine and/or with other
                        % versions of LaTeX. The latest version and
                        % documentation can be obtained at:
                        % http://www.ctan.org/tex-archive/macros/latex/contrib/supported/sttools/
                        % Documentation is contained in the stfloats.sty
                        % comments as well as in the presfull.pdf file.
                        % Do not use the stfloats baselinefloat ability as
                        % IEEE does not allow \baselineskip to stretch.
                        % Authors submitting work to the IEEE should note
                        % that IEEE rarely uses double column equations and
                        % that authors should try to avoid such use.
                        % Do not be tempted to use the cuted.sty or
                        % midfloat.sty package (by the same author) as IEEE
                        % does not format its papers in such ways.

\usepackage{amsmath}    % From the American Mathematical Society
                        % A popular package that provides many helpful commands
                        % for dealing with mathematics. Note that the AMSmath
                        % package sets \interdisplaylinepenalty to 10000 thus
                        % preventing page breaks from occurring within multiline
                        % equations. Use:
%\interdisplaylinepenalty=2500
                        % after loading amsmath to restore such page breaks
                        % as IEEEtran.cls normally does. amsmath.sty is already
                        % installed on most LaTeX systems. The latest version
                        % and documentation can be obtained at:
                        % http://www.ctan.org/tex-archive/macros/latex/required/amslatex/math/



% Other popular packages for formatting tables and equations include:

%\usepackage{array}
% Frank Mittelbach's and David Carlisle's array.sty which improves the
% LaTeX2e array and tabular environments to provide better appearances and
% additional user controls. array.sty is already installed on most systems.
% The latest version and documentation can be obtained at:
% http://www.ctan.org/tex-archive/macros/latex/required/tools/

% V1.6 of IEEEtran contains the IEEEeqnarray family of commands that can
% be used to generate multiline equations as well as matrices, tables, etc.

% Also of notable interest:
% Scott Pakin's eqparbox package for creating (automatically sized) equal
% width boxes. Available:
% http://www.ctan.org/tex-archive/macros/latex/contrib/supported/eqparbox/

% *** Do not adjust lengths that control margins, column widths, etc. ***
% *** Do not use packages that alter fonts (such as pslatex).         ***
% There should be no need to do such things with IEEEtran.cls V1.6 and later.


% Your document starts here!
\begin{document}

% Define document title and author
	\title{Senior Projekt Labor Notizbuch}
	\author{John Knowles
	\thanks{Advisors: Professor Tom Bensky tbensky@calpoly.edu, Professor Matt Moelter mmoelter@calpoly.edu}}
	\markboth{Senior Project Lab Notebook}{}
	\maketitle

% Write abstract here
\begin{abstract}
	Recorded Notes of the process, findings, and development of this project.
\end{abstract}

% Each section begins with a \section{title} command
%\setcounter{subsubsection}{3}
\section{Time Log}
	% \PARstart{}{} creates a tall first letter for this first paragraph
%\subsubsection{1/29/16}
\begin{itemize}
	\item 1.29.16:
	\begin{itemize}
    	\item Created this lab notebook, bought some necessary parts, including camera, camera mount, and a raspberry pi case.
        \item Booted clean copy of Raspian (Wheezy), installed the following:
        \begin{itemize}
			\item fswebcam; 3rd party webcam pkg.
            \item tesseract-ocr; Optical Character Recognition pkg.
            \item Successfully took images with raspberry pi; however, uploading it to tesseract for processing returned blank space. 
            \item Decided to try another route. Installed Mobaxterm to SSH into raspberry pi and then updated everything:
            \begin{lstlisting}[language=bash]
$ sudo apt-get update
$ sudo apt-get update
$ sudo apt-get upgrade
$ sudo rpi-update
			\end{lstlisting}
            \item Installed required developer tools and packages for OpenCV:
             \begin{lstlisting}[language=bash]
$ sudo apt-get install build-essential cmake pkg-config
			\end{lstlisting}
            \item Installed the necessary image I/O packages. Allows you to load various image file formats such as JPEG, PNG, TIFF:
            \begin{lstlisting}[language=bash]
$ sudo apt-get install libjpeg8-dev libtiff4-dev libjasper-dev libpng12-dev
			\end{lstlisting}
            \item Installed the GTK development library:
            \begin{lstlisting}[language=bash]
$ sudo apt-get install libgtk2.0-dev
			\end{lstlisting}
            \item Installed the necessary video I/O packages for OpenCV:
            \begin{lstlisting}[language=bash]
$ sudo apt-get install libavcodec-dev libavformat-dev libswscale-dev libv4l-dev
			\end{lstlisting}
            \item Installed libraries that are used to optimize various operations within OpenCV:
            \begin{lstlisting}[language=bash]
$ sudo apt-get install libatlas-base-dev gfortran
			\end{lstlisting}
            \item Pulled the OpenCV repository from GitHub and checkout the 3.1.0  version:
            \begin{lstlisting}[language=bash]
$ cd ~
$ git clone https://github.com/Itseez/opencv.git
$ cd opencv
$ git checkout 3.1.0
			\end{lstlisting}
            \item Since I would be working with Python 3+, I compiled OpenCV for Python 3+ support. I chose 3+ over 2.7 since it was unlikely that I would need the support of scientific packages not up-to-date with Python 3+. 
            \item Installed the Python 3+ headers.
            \begin{lstlisting}[language=bash]
$ sudo apt-get install python3-dev
			\end{lstlisting}
            \item Installed pip for Python 3 compatibility.
            \begin{lstlisting}[language=bash]
$ wget https://bootstrap.pypa.io/get-pip.py
$ sudo python3 get-pip.py
			\end{lstlisting}
            \item Installed virtualenv and virtualenvwrapper:
            \begin{lstlisting}[language=bash]
$ sudo pip3 install virtualenv virtualenvwrapper
			\end{lstlisting}
            \item Updated \texttildelow /.profile
            \begin{lstlisting}[language=bash]
# virtualenv and virtualenvwrapper
export VIRTUALENVWRAPPER_PYTHON=/usr/bin/python3
export WORKON_HOME=$HOME/.virtualenvs
source /usr/local/bin/virtualenvwrapper.sh
			\end{lstlisting}
            \item Commit changes to \texttildelow /.profile
            \begin{lstlisting}[language=bash]
$ source ~/.profile
			\end{lstlisting}
            \item Made a CV virtual environment:
            \begin{lstlisting}[language=bash]
$ mkvirtualenv cv
			\end{lstlisting}
            \item To access the virtual environment:
            \begin{lstlisting}[language=bash]
$ workon cv
			\end{lstlisting}
            \item Made sure I had Numpy:
            \begin{lstlisting}[language=bash]
$ pip install numpy
			\end{lstlisting}
            \item On the off-chance there were permission errors:
            \begin{lstlisting}[language=bash]
$ sudo rm -rf ~/.cache/pip/
$ pip install numpy
			\end{lstlisting}
            \item The following actions require the CV virtual environment, otherwise "ImportError: No module named cv2" will show up:
            \begin{lstlisting}[language=bash]
$ cd ~/opencv
$ mkdir build
$ cd build
$ cmake -D CMAKE_BUILD_TYPE=RELEASE \
	-D CMAKE_INSTALL_PREFIX=/usr/local \
	-D INSTALL_C_EXAMPLES=OFF \
	-D INSTALL_PYTHON_EXAMPLES=ON \
	-D OPENCV_EXTRA_MODULES_PATH=~/opencv_contrib/modules \
	-D BUILD_EXAMPLES=ON ..
			\end{lstlisting}
            \item Made sure CMake picked up Python 3 interpreter and points to CV environment for both python and numpy. Output should read something like "Python 3 - Interpreter - /home/pi/.virtualenvs/cv/bin/python3 (ver. 3.*.*)..."
            \item Compiled OpenCV 3.0
            \begin{lstlisting}[language=bash]
$ make -j4
			\end{lstlisting}
            \item After compilation, I installed it.
            \begin{lstlisting}[language=bash]
$ sudo make install
$ sudo ldconfig
			\end{lstlisting}
            \item Checked pkg path:
            \begin{lstlisting}[language=bash]
$ ls -l /usr/local/lib/python3.2/site-packages
total 1416
-rw-r--r-- 1 root staff 1447637 Jun 22 18:26 cv2.cpython-32mu.so
			\end{lstlisting}
            \item Created a sym-link to OpenCV binary into the site-packages directory of the CV environment:
            \begin{lstlisting}[language=bash]
$ cd ~/.virtualenvs/cv/lib/python3.2/site-packages/
$ ln -s /usr/local/lib/python3.2/site-packages/cv2.cpython-32mu.so cv2.so
			\end{lstlisting}
            \item While still in the CV virtual environment, I verified the install was successful:
            \begin{lstlisting}[language=bash]
$ workon cv
$ python
>>> import cv2
>>> cv2.__version__
'3.0.0'
			\end{lstlisting}
            
		\end{itemize}
	\end{itemize}
\end{itemize}


















%	\PARstart{T}{his} section introduces the topic and leads the reader on to the main part.

% Main Part
%\section{Main Part}
	% LaTeX takes complete care of your document layout ...
%	The presentation's content is summarized in the report in 4~pages.
	% ... but you can insert a line break manually with two backslashes, if needed: \\


%	\begin{table}[!hbt]
%		% Center the table
%		\begin{center}
		% Title of the table
%		\caption{Simulation Parameters}
%		\label{tab:simParameters}
%		% Table itself: here we have two columns which are centered and have lines to the left, right and in the middle: |c|c|
%		\begin{tabular}{|c|c|}
%			% To create a horizontal line, type \hline
%			\hline
%			% To end a column type &
%			% For a linebreak type \\
%			Information message length & $k=16000$ bit \\
%			\hline
%			Radio segment size & $b=160$ bit \\
%			\hline
%			Rate of component codes & $R_{cc}=1/3$\\
%			\hline
%			Polynomial of component encoders & $[1 , 33/37 , 25/37]_8$\\
%			\hline
%		\end{tabular}
%		\end{center}
%	\end{table}

	% If you have questions about how to write mathematical formulas in LaTeX, please read a LaTeX book or the 'Not So Short Introduction to LaTeX': tobi.oetiker.ch/lshort/lshort.pdf

	% This is how you include a eps figure in your document. LaTeX only accepts EPS or TIFF files.
%	\begin{figure}[!hbt]
%		% Center the figure.
%		\begin{center}
%		% Include the eps file, scale it such that it's width equals the column width. You can also put width=8cm for example...
%		\includegraphics[width=\columnwidth]{plot_tf}
%		% Create a subtitle for the figure.
%		\caption{Simulation results on the AWGN channel. Average throughput $k/n$ vs $E_s/N_0$.}
%		% Define the label of the figure. It's good to use 'fig:title', so you know that the label belongs to a figure.
%		\label{fig:tf_plot}
%		\end{center}
%	\end{figure}

\section{Filling this page}
\section{Conclusion}

% Now we need a bibliography:
%\begin{thebibliography}{5}

	%Each item starts with a \bibitem{reference} command and the details thereafter.
%	\bibitem{HOP96} % Transaction paper
%	J.~Hagenauer, E.~Offer, and L.~Papke. Iterative decoding of binary block
%	and convolutional codes. {\em IEEE Trans. Inform. Theory},
%	vol.~42, no.~2, pp.~429–-445, Mar. 1996.
%
%	\bibitem{MJH06} % Conference paper
%	T.~Mayer, H.~Jenkac, and J.~Hagenauer. Turbo base-station cooperation for intercell interference cancellation. {\em IEEE Int. Conf. Commun. (ICC)}, Istanbul, Turkey, pp.~356--361, June 2006.

%	\bibitem{Proakis} % Book
%	J.~G.~Proakis. {\em Digital Communications}. McGraw-Hill Book Co.,
%	New York, USA, 3rd edition, 1995.
%
%	\bibitem{talk} % Web document
%	F.~R.~Kschischang. Giving a talk: Guidelines for the Preparation and Presentation of Technical Seminars.
%	\url{http://www.comm.toronto.edu/frank/guide/guide.pdf}.
%
%	\bibitem{5}
%	IEEE Transactions \LaTeX and Microsoft Word Style Files.
%	\url{http://www.ieee.org/web/publications/authors/transjnl/index.html}
%\end{thebibliography}

% Your document ends here!
\end{document}